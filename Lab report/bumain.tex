\documentclass{article}
\usepackage{graphicx} % Required for inserting images
\usepackage{amsmath}
\usepackage{listings}
\usepackage{xcolor}
\lstset { %
    language=C++,
    backgroundcolor=\color{black!5}, % set backgroundcolor
    basicstyle=\footnotesize,% basic font setting
}
\title{DFT Implementation Using STO-3G Basis Sets: 
A Computational Approach to Electronic Structure Calculations}
\author{Magane Tamandja}
\date{\today}

\begin{document}

\maketitle

\section{abstract}
This report presents an efficient implementation of Density Functional Theory (DFT) using the STO-3G basis set for electronic structure calculations. We develop analytical expressions for the Laplacian operator,
 electron repulsion terms, and exchange-correlation potentials within the Local Density Approximation (LDA) framework. Our implementation demonstrates [briefly mention a key result or advantage of your approach].
 The mathematical framework established here provides a foundation for more advanced implementations that could significantly reduce computational costs while maintaining accuracy for molecular systems.

\section{Introduction}
1. Introduction

Density Functional Theory (DFT) has emerged as one of the most powerful and widely used methods for electronic structure calculations in computational chemistry and materials science. The popularity of 
DFT stems from its favorable balance between computational efficiency and accuracy, making it suitable for studying systems ranging from small molecules to extended materials. 

Despite its widespread use, implementing DFT algorithms that are both accurate and computationally efficient remains challenging. The treatment of exchange-correlation functionals,
 basis set selection, and numerical integration schemes all significantly impact the performance and reliability of DFT calculations.

In this report, we present a detailed implementation of a DFT algorithm using the STO-3G basis set. We focus specifically on three critical components:
1) The mathematical treatment of the Laplacian operator
2) The computation of electron repulsion integrals
3) The implementation of exchange-correlation potentials within the Local Density Approximation (LDA)

Our goal is to develop a computational framework that balances accuracy with efficiency for electronic structure calculations, providing a foundation for future extensions to more complex molecular systems.
\section{The Khon-Sham equation}

In the following sections we will develop the algorithms given by the KS equation given by : 
\[\left(-\frac{1}{2} \nabla^2 + \left[\sum_{j}^{N}\int \frac{|\varphi_j(\vec{r_2})|^2}{r_{12}}d\vec{r_2} + V_{XC}(\vec{r_1})- \sum_{A}^{M}\frac{Z_A}{r_{1A}} \right]\right)\varphi_i = \epsilon_i\varphi_i\]
The strategy is to get to an eigen value problem. So we have to  transform the expression
 $ \left(-\frac{1}{2} \nabla^2 + \left[\sum_{j}^{N}\int \frac{|\varphi_j(\vec{r_2})|^2}{r_{12}}d\vec{r_2} + V_{XC}(\vec{r_1})- \sum_{A}^{M}\frac{Z_A}{r_{1A}} \right]\right)$ 
 into a matrix operator.
\subsection{Treatment of the Laplacian}
Given the Laplacian operator $-\frac{1}{2}\nabla^2$ ,we choose to use the STO-3G basis set, which is given by : 
\begin{equation*}
    \psi_{\text{STO-3G}}(s) = c_1 \phi_1 + c_2 \phi_2 + c_3 \phi_3
    \end{equation*}

where

    \begin{align*}
    \phi_1 &= \left( \frac{2\alpha_1}{\pi} \right)^{3/4} e^{-\alpha_1 r^2} \\
    \phi_2 &= \left( \frac{2\alpha_2}{\pi} \right)^{3/4} e^{-\alpha_2 r^2} \\
    \phi_3 &= \left( \frac{2\alpha_3}{\pi} \right)^{3/4} e^{-\alpha_3 r^2}
    \end{align*}

So we can rewrite the Laplacian as 

\begin{align*}
    \frac{d^2(c_1 \phi_1 + c_2 \phi_2 + c_3 \phi_3)}{d^2r} &= \frac{d^2(c_1  \left( \frac{2\alpha_1}{\pi} \right)^{3/4} e^{-\alpha_1 r^2} + c_2 \left( \frac{2\alpha_2}{\pi} \right)^{3/4} e^{-\alpha_2 r^2} + c_3 \left( \frac{2\alpha_3}{\pi} \right)^{3/4} e^{-\alpha_3 r^2})}{d^2r} \\
                                                           &= (4\alpha_1^2r^2 - 2\alpha_1)c_1  \left( \frac{2\alpha_1}{\pi} \right)^{3/4} e^{-\alpha_1 r^2} \\
                                                           &+(4\alpha_2^2r^2 - 2\alpha_2)c_2  \left( \frac{2\alpha_2}{\pi} \right)^{3/4} e^{-\alpha_2 r^2}  \\
                                                           &+(4\alpha_3^2r^2 - 2\alpha_3)c_3  \left( \frac{2\alpha_3}{\pi} \right)^{3/4} e^{-\alpha_3 r^2}
\end{align*}
In the spirit of constructing a matrix hamiltonian, we diagonilize the expression to get the expression \\
 \(\frac{d^2(c_1 \phi_1 + c_2 \phi_2 + c_3 \phi_3)}{d^2r}\) = \\
$\begin{bmatrix}
    (4\alpha_1^2r^2 - 2\alpha_1) & 0 & 0\\
    0 & (4\alpha_2^2r^2 - 2\alpha_2) & 0 \\
    0& 0& (4\alpha_3^2r^2 - 2\alpha_3)
\end{bmatrix} (c_1  \left( \frac{2\alpha_1}{\pi} \right)^{3/4} e^{-\alpha_1 r^2} + c_2 \left( \frac{2\alpha_2}{\pi} \right)^{3/4} e^{-\alpha_2 r^2} + c_3 \left( \frac{2\alpha_3}{\pi} \right)^{3/4} e^{-\alpha_3 r^2}).$

This provides a computable  expression for the Laplacian. We still run into the elements of the of the matrix having the variable r. We therefore apply the formula: \[L_{ij} = \int \phi^*_i(r) (-\frac{1}{2} \nabla^2 )\phi_j(r)dr. \]
In our case, we have :
\begin{align*}
    L_{ij} &= \int \phi^*_i(r) (-\frac{1}{2} \nabla^2 )\phi_j(r)dr \\
            &= \int \phi^*_i(r) (4\alpha_i^2r^2 - 2\alpha_i) \phi_j(r)dr , \text{since we are working with a diagonal matrix and STO -3G is real,}\\
            &= \int \phi_i(r) (4\alpha_i^2r^2 - 2\alpha_i) \phi_i(r)dr \\
            &= \int \phi_i(r)^2(4\alpha_i^2r^2 - 2\alpha_i)\\
            &= \int \left(  \left( \frac{2\alpha_i}{\pi} \right)^{3/4}e^{-\alpha_i r^2}\right)^2(4\alpha_i^2r^2 - 2\alpha_i)\\
            &= \int \left(  \left( \frac{2\alpha_i}{\pi} \right)^{3/2}e^{-2\alpha_i r^2}\right)(4\alpha_i^2r^2 - 2\alpha_i)\\
            &= \int \left( 4\alpha_i^2 \left( \frac{2\alpha_i}{\pi} \right)^{3/2}r^2e^{-2\alpha_i r^2} - 2\alpha_i \left( \frac{2\alpha_i}{\pi} \right)^{3/2}e^{-2\alpha_i r^2} \right)\\
            &= \int  4\alpha_i^2 \left( \frac{2\alpha_i}{\pi} \right)^{3/2}r^2e^{-2\alpha_i r^2} - \int 2\alpha_i \left( \frac{2\alpha_i}{\pi} \right)^{3/2}e^{-2\alpha_i r^2} \\
            &= 4\alpha_i^2 \left( \frac{2\alpha_i}{\pi} \right)^{3/2} \frac{1}{4\alpha_i} \sqrt{ \frac{\pi}{2\alpha_i}} - 2\alpha_i \left( \frac{2\alpha_i}{\pi} \right)^{3/2}\sqrt{\frac{\pi}{2\alpha_i}}\\
        L_{ii}    &= - \alpha_i \left( \frac{2\alpha_i}{\pi} \right)^{3/2}\sqrt{\frac{\pi}{2\alpha_i}}.
\end{align*}
Finally, we get the Laplacian :
\\
\\
$\begin{bmatrix}
    - \alpha_1 \left( \frac{2\alpha_1}{\pi} \right)^{3/2}\sqrt{\frac{\pi}{2\alpha_1}} & 0 & 0\\
    0 & - \alpha_2 \left( \frac{2\alpha_2}{\pi} \right)^{3/2}\sqrt{\frac{\pi}{2\alpha_2}} & 0 \\
    0& 0& - \alpha_3 \left( \frac{2\alpha_3}{\pi} \right)^{3/2}\sqrt{\frac{\pi}{2\alpha_3}}
\end{bmatrix}$
\\
The following C++ code is a function implementation that is needed to generate each element:

\begin{lstlisting}
        double y_feta(double c , double a){
            return c*((2*a)/M_PI)^(3/4);
        }

        double y_gaussian_integral(double a, double b){
            return sqrt(M_PI/(a+b));
        }
\end{lstlisting}

\subsection{Treatment of the Electron Repulsion Term}
We start by the following assumption, the integral is defined by infinity, this make calculations convenient.\\
Given the term \( |\varphi_j(\vec{r_2})|^2\) can be written as \( (\varphi_j(\vec{r_2}))^2\) for a real basis function, we write: 
\begin{align*}
    |\varphi_j(\vec{r_2})|^2  &= \left[ c_1  \left( \frac{2\alpha_1}{\pi} \right)^{3/4} e^{-\alpha_1 r^2} + c_2 \left( \frac{2\alpha_2}{\pi} \right)^{3/4} e^{-\alpha_2 r^2} + c_3 \left( \frac{2\alpha_3}{\pi} \right)^{3/4} e^{-\alpha_3 r^2} \right] ^2 \\
                             &= \left[ c_1  \left( \frac{2\alpha_1}{\pi} \right)^{3/4} \right] ^2 e^{-2\alpha_1 r^2} + 2 \left[ c_2 \left( \frac{2\alpha_2}{\pi} \right)^{3/4} \right] \left[ c_1  \left( \frac{2\alpha_1}{\pi} \right)^{3/4} \right] e^{-\alpha_1 r^2 - \alpha_2 r^2} \\
                             &+ 2\left[ c_3 \left( \frac{2\alpha_3}{\pi} \right)^{3/4} \right] \left[ c_1  \left( \frac{2\alpha_1}{\pi} \right)^{3/4} \right] e^{-\alpha_1 r^2 - \alpha_3 r^2} + \left[ c_2  \left( \frac{2\alpha_2}{\pi} \right)^{3/4} \right] ^2 e^{-2\alpha_2 r^2} \\ 
                             &+ \left[ c_3  \left( \frac{2\alpha_3}{\pi} \right)^{3/4} \right] ^2 e^{-2\alpha_3 r^2} + 2 \left[ c_2 \left( \frac{2\alpha_2}{\pi} \right)^{3/4} \right] \left[ c_3  \left( \frac{2\alpha_3}{\pi} \right)^{3/4} \right] e^{-\alpha_2 r^2 - \alpha_3 r^2} \\
\end{align*}
\begin{align*}                             
    |\varphi_j(\vec{r_2})|^2                         &= \left[ c_1  \left( \frac{2\alpha_1}{\pi} \right)^{3/4} \right] ^2 e^{-2\alpha_1 r^2} + 2 \left[ c_2 \left( \frac{2\alpha_2}{\pi} \right)^{3/4} \right] \left[ c_1  \left( \frac{2\alpha_1}{\pi} \right)^{3/4} \right] e^{(-\alpha_1 - \alpha_2 )r^2} \\
                             &+ 2\left[ c_3 \left( \frac{2\alpha_3}{\pi} \right)^{3/4} \right] \left[ c_1  \left( \frac{2\alpha_1}{\pi} \right)^{3/4} \right] e^{(-\alpha_1  - \alpha_3 )r^2} + \left[ c_2  \left( \frac{2\alpha_2}{\pi} \right)^{3/4} \right] ^2 e^{-2\alpha_2 r^2} \\ 
                             &+ \left[ c_3  \left( \frac{2\alpha_3}{\pi} \right)^{3/4} \right] ^2 e^{-2\alpha_3 r^2} + 2 \left[ c_2 \left( \frac{2\alpha_2}{\pi} \right)^{3/4} \right] \left[ c_3  \left( \frac{2\alpha_3}{\pi} \right)^{3/4} \right] e^{(-\alpha_2 - \alpha_3 )r^2} \\
\end{align*}
\begin{align*} 
                             \int |\varphi_j(\vec{r_2})|^2dr                         &= \int \left[ c_1  \left( \frac{2\alpha_1}{\pi} \right)^{3/4} \right] ^2 e^{-2\alpha_1 r^2}dr + \int 2 \left[ c_2 \left( \frac{2\alpha_2}{\pi} \right)^{3/4} \right] \left[ c_1  \left( \frac{2\alpha_1}{\pi} \right)^{3/4} \right] e^{(-\alpha_1 - \alpha_2 )r^2}dr \\
                             &+ \int 2\left[ c_3 \left( \frac{2\alpha_3}{\pi} \right)^{3/4} \right] \left[ c_1  \left( \frac{2\alpha_1}{\pi} \right)^{3/4} \right] e^{(-\alpha_1  - \alpha_3 )r^2}dr + \int \left[ c_2  \left( \frac{2\alpha_2}{\pi} \right)^{3/4} \right] ^2 e^{-2\alpha_2 r^2}dr \\ 
                             &+ \int \left[ c_3  \left( \frac{2\alpha_3}{\pi} \right)^{3/4} \right] ^2 e^{-2\alpha_3 r^2}dr + \int 2 \left[ c_2 \left( \frac{2\alpha_2}{\pi} \right)^{3/4} \right] \left[ c_3  \left( \frac{2\alpha_3}{\pi} \right)^{3/4} \right] e^{(-\alpha_2 - \alpha_3 )r^2}dr \\
   \int |\varphi_j(\vec{r_2})|^2  dr                       &= \left[ c_1  \left( \frac{2\alpha_1}{\pi} \right)^{3/4} \right] ^2 \int e^{-2\alpha_1 r^2}dr + 2 \left[ c_2 \left( \frac{2\alpha_2}{\pi} \right)^{3/4} \right] \left[ c_1  \left( \frac{2\alpha_1}{\pi} \right)^{3/4} \right] \int e^{(-\alpha_1 - \alpha_2 )r^2}dr \\
                             &+ 2\left[ c_3 \left( \frac{2\alpha_3}{\pi} \right)^{3/4} \right] \left[ c_1  \left( \frac{2\alpha_1}{\pi} \right)^{3/4} \right] \int e^{(-\alpha_1  - \alpha_3 )r^2}dr + \left[ c_2  \left( \frac{2\alpha_2}{\pi} \right)^{3/4} \right] ^2 \int e^{-2\alpha_2 r^2}dr \\ 
                             &+ \left[ c_3  \left( \frac{2\alpha_3}{\pi} \right)^{3/4} \right] ^2 \int e^{-2\alpha_3 r^2}dr + 2 \left[ c_2 \left( \frac{2\alpha_2}{\pi} \right)^{3/4} \right] \left[ c_3  \left( \frac{2\alpha_3}{\pi} \right)^{3/4} \right] \int e^{(-\alpha_2 - \alpha_3 )r^2}dr \\
\end{align*}
   Using the Gaussian integral formula:  \[\int_{-\infty}^{\infty} e^{-\alpha r^2}dr = \sqrt{\frac{\pi}{\alpha}};\] we can rewrite the integral terms as : 

   \begin{align*}
    \int |\varphi_j(\vec{r_2})|^2 dr                        &= \left[ c_1  \left( \frac{2\alpha_1}{\pi} \right)^{3/4} \right] ^2 \sqrt{\frac{\pi}{2\alpha_1}}  + 2 \left[ c_2 \left( \frac{2\alpha_2}{\pi} \right)^{3/4} \right] \left[ c_1  \left( \frac{2\alpha_1}{\pi} \right)^{3/4} \right] \sqrt{\frac{\pi}{\alpha_1+ \alpha_2}}\\
    &+ 2\left[ c_3 \left( \frac{2\alpha_3}{\pi} \right)^{3/4} \right] \left[ c_1  \left( \frac{2\alpha_1}{\pi} \right)^{3/4} \right] \sqrt{\frac{\pi}{\alpha_1 + \alpha_2}} + \left[ c_2  \left( \frac{2\alpha_2}{\pi} \right)^{3/4} \right] ^2 \sqrt{\frac{\pi}{2\alpha_2}} \\ 
    &+ \left[ c_3  \left( \frac{2\alpha_3}{\pi} \right)^{3/4} \right] ^2 \sqrt{\frac{\pi}{2\alpha_3}}  + 2 \left[ c_2 \left( \frac{2\alpha_2}{\pi} \right)^{3/4} \right] \left[ c_3  \left( \frac{2\alpha_3}{\pi} \right)^{3/4} \right] \sqrt{\frac{\pi}{\alpha_2 + \alpha_3}}  \\
   \end{align*}

This provides us with a relatively easy to compute a coulomb contribution. Using the point charge approximation.
   \[V_c \approx \frac{Q_AQ_B}{R_{AB}}. \]
   With $Q_A \approx \int |\phi_A(r)|^2dr $, where $\phi$ is $\varphi$ for all $c_i = 1$. A and B are nucleus centers. 
We implement the following function to perform these calculations:
   \begin{lstlisting}
    double y_coulomb(double feta_1, double feta_2, double feta_3, 
    double gi1, double gi2, double gi3, double gi4, double gi5){
     return (((feta_1^2)*gi1)+(2*feta_2*feta_1*gi2)+(2*feta_3*
     feta_1*gi2)+((feta_2^2)*gi3)+((feta_3^2)*gi4)+
     (2*feta_2*feta_3*gi5));
    }

    double y_coulomb_interaction(double QA, double QB, double RAB){
    return (QA*QB)/RAB;

    }

   \end{lstlisting}
\subsection{Treatment of the exchange potential : LDA}
Using the LDA approximation of the exchange potential, we get:
\begin{align*} 
    V^{LDA}_X  &= -(\frac{3}{\pi})^{\frac{1}{3}} \varphi^{\frac{1}{3}}(r) \\
    V^{LDA}_X   &= -(\frac{3}{\pi})\sum_{j}^{occ}|\varphi_j(r)|^2 \\
\end{align*}




 

\end{document}
